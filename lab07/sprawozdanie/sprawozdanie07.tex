\documentclass[a4paper]{article}

%% Language and font encodings
\usepackage{polski}
\usepackage[polish]{babel}
\usepackage[utf8x]{inputenc}
\usepackage[T1]{fontenc}
\usepackage{pdfpages}
\usepackage{indentfirst}
\usepackage{listings}
\usepackage{isotope}

\usepackage{csvsimple}
\setlength{\tabcolsep}{2pt}

% Adjust penalties
\brokenpenalty=1000
\clubpenalty=1000
\widowpenalty=1000

% Don't break in math expressions
\relpenalty=10000
\binoppenalty=10000

%% Sets page size and margins
\usepackage[a4paper]{geometry}

%% Useful packages
\usepackage{amsmath}
\usepackage{graphicx}
\usepackage[colorlinks=true, allcolors=blue]{hyperref}
\usepackage{booktabs}
\usepackage{cancel}
\usepackage{tikz}

\usepackage{float}

\renewcommand\thesection{\arabic{section}.}
\renewcommand\thesubsection{\arabic{section}.\arabic{subsection}.}
\renewcommand\thesubsubsection{\arabic{section}.\arabic{subsection}.\arabic{subsubsection}.}

% The following commands are not supported in PSTricks at present
% We define them conditionally, so when they are implemented,
% this pgf file will use them.
\ifx\du\undefined
  \newlength{\du}
\fi
\setlength{\du}{15\unitlength}

\newcommand{\Vsp}[1]{\vtop to #1 {}}
\newcommand{\Hsp}[1]{\hbox to #1 {}}
\newcommand{\Small}{\scriptsize}

\title{Sprawozdanie nr 7}
\date{}


\begin{document}

\begin{center}
\begin{tabular}{|p{5.5cm}|l|l|c|}
    \hline
	% Row 1.1  
	    Wydział \Vsp{4mm} &
	    \multicolumn{1}{|l}{Dzień} &
	    poniedziałek $17^{15} - 19^{30}$ &
	    Nr zespołu \\
	% Row 1.2
	    \mbox{\small{Matematyki i Nauk Informatycznych}} &
	    \multicolumn{1}{|l}{Data}  &
	    &
	    \multicolumn{1}{c|}{\Large{18}} \\
    
    \hline
	% Row 2.1 
	    Nazwisko i Imię: &
	    \Small Ocena z przygotowania &
	    \Small Ocena ze sprawozdania &
	    \Small Ocena Końcowa \\
	% Rows 2.2-2.4
	    1. Jasiński Bartosz & & &\\
	    2. Sadłocha Adrian & & & \\
	    3. Wódkiewicz Andrzej & & & \\

    \hline
    % Row 3.1
	    \multicolumn{2}{|l|}{Prowadzący \Vsp{4mm}} &
	    \multicolumn{2}{|l|}{Podpis prowadzącego} \\  
    % Row 3.2
    	\multicolumn{2}{|l|}{dr hab. Katarzyna Grebieszkow} &
    	\multicolumn{2}{|l|}{} \\    	
    \hline
\end{tabular}
\label{pieczatka}
\end{center}

{\let\newpage\relax\maketitle}
\setcounter{secnumdepth}{2}


\section{Opis ćwiczenia}

\subsection{Wstęp teoretyczny}

\subsection{Układ pomiarowy}

\section{Pomiary i wstępne obliczenia}

Dokonaliśmy dwóch serii pomiarów, kolejno $A$ oraz $B$.
W obu seriach napięcie (ozn. $U$) ustalane było między ok. $3.50$ kV a ok. $11.10$ kV, z różnicą ok. $0.40$ kV pomiędzy pomiarami.
Podczas serii $A$ napięcie rosło wraz z każdym pomiarem, podczas serii $B$ -- malało.

Ponieważ stan podczas odczytu napięcia nie był stabilny (największy zauważony przez nas skok wartości wyniósł $0.09$ kV), za niepewność całkowitą pomiaru $U$ przyjęliśmy:

$$u(U) = \frac{0.09 \, \text{kV}}{\sqrt 3}$$

Dla ustalonego napięcia interesowała nas średnica najbardziej wewnętrznego pierścienia, ozn.~$d$.
W celu wyznaczenia tej wartości, mierzona była średnica wewnętrzna oraz zewnętrzna pierścienia, oznaczane kolejno $d_w$ oraz~$d_z$.
Za średnicę właściwą przyjęliśmy średnią arytmetyczną z powyższych wartości, tj.~$d = \frac {d_w + d_z}{2}$.

Użyty przyrząd pomiarowy miał podziałkę wynoszącą $\Delta d_w = 1$ mm.
Ze względu na rozmycie pierścieni, za niepewność eksperymentatora przyjęliśmy dwukrotność podziałki: $\Delta d_{w_E} = 2 \cdot \Delta d_w = 2$~mm.
Zatem niepewność całkowita pomiaru średnicy wewnętrznej (a także zewnętrznej, którą oznaczymy przez $u(d_z)$) wynosi:

$$u(d_w) = \sqrt{\frac{\Delta d_w^2}{3} + \frac{\Delta d_{w_E}^2}{3}} = \sqrt \frac{5}{3} \Delta d_w = \sqrt \frac{5}{3} \, \text{mm}$$

Nas jednak interesuje niepewność średnicy właściwej.
Skoro $d = \frac {d_w + d_z}{2}$, to -- korzystając z prawa propagacji niepewności -- niepewność całkowita pomiaru średnicy wynosi:

$$u(d) = \sqrt{\left(\frac{\partial d}{\partial d_w}\right)^2 \cdot u^2(d_w) + \left(\frac{\partial d}{\partial d_w}\right)^2 \cdot u^2(d_z)} = \sqrt{\frac{1}{4} u^2(d_w) + \frac{1}{4} u^2(d_z)} = \frac{u(d_w)}{\sqrt{2}}$$

Po podstawieniu wyliczonej wcześniej niepewności otrzymujemy:

$$u(d) = \sqrt \frac{5}{6} \, \text{mm} \approx 0.912870929175 \, \text{mm} \approx 1 \, \text{mm}$$

\section{Opracowanie wyników}

\subsection{Zgodność wyników z teorią}

\subsection{Odległości między płaszczyznami atomowymi}

\subsection{Wyliczenie odległości dla pozostałych pierścieni}

\section{Wnioski}

\subsection{Prawdziwość hipotezy de Broglie'a}

\end{document}
