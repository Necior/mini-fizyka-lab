\documentclass[a4paper]{article}

%% Language and font encodings
\usepackage{polski}
\usepackage[utf8]{inputenc}
\usepackage[T1]{fontenc}
\usepackage{pdfpages}
\usepackage{indentfirst}

% Adjust penalties
\brokenpenalty=1000
\clubpenalty=1000
\widowpenalty=1000

%% Sets page size and margins
\usepackage[a4paper]{geometry}

%% Useful packages
\usepackage{amsmath}
\usepackage{graphicx}
\usepackage[colorlinks=true, allcolors=blue]{hyperref}
\usepackage{booktabs}
\usepackage{cancel}
\usepackage{tikz}

\usepackage{float}

\renewcommand\thesection{\arabic{section}.}
\renewcommand\thesubsection{\arabic{section}.\arabic{subsection}.}
\renewcommand\thesubsubsection{\arabic{section}.\arabic{subsection}. \arabic{subsubsection}.}

% The following commands are not supported in PSTricks at present
% We define them conditionally, so when they are implemented,
% this pgf file will use them.
\ifx\du\undefined
  \newlength{\du}
\fi
\setlength{\du}{15\unitlength}

\newcommand{\Vsp}[1]{\vtop to #1 {}}
\newcommand{\Hsp}[1]{\hbox to #1 {}}
\newcommand{\Small}{\scriptsize}

\title{Sprawozdanie nr 2}
\date{}


\begin{document}

\begin{center}
\begin{tabular}{|p{5.5cm}|l|l|c|}
    \hline
	% Row 1.1  
	    Wydział \Vsp{4mm} &
	    \multicolumn{1}{|l}{Dzień} &
	    poniedziałek $17^{15} - 19^{30}$ &
	    Nr zespołu \\
	% Row 1.2
	    \mbox{\small{Matematyki i Nauk Informatycznych}} &
	    \multicolumn{1}{|l}{Data}  &
	    &
	    \multicolumn{1}{c|}{\Large{18}} \\
    
    \hline
	% Row 2.1 
	    Nazwisko i Imię: &
	    \Small Ocena z przygotowania &
	    \Small Ocena ze sprawozdania &
	    \Small Ocena Końcowa \\
	% Rows 2.2-2.4
	    1. Jasiński Bartosz & & &\\
	    2. Sadłocha Adrian & & & \\
	    3. Wódkiewicz Andrzej & & & \\

    \hline
    % Row 3.1
	    \multicolumn{2}{|l|}{Prowadzący \Vsp{4mm}} &
	    \multicolumn{2}{|l|}{Podpis prowadzącego} \\  
    % Row 3.2
    	\multicolumn{2}{|l|}{dr inż. Jarosław Judyk} &
    	\multicolumn{2}{|l|}{} \\    	
    \hline
\end{tabular}
\label{pieczatka}
\end{center}

{\let\newpage\relax\maketitle}  % stolen from: https://tex.stackexchange.com/questions/86249/maketitle-text-before-title
\setcounter{secnumdepth}{2}


\section{Opis ćwiczenia}
Ćwiczenie złożone było z dwóch części:
\begin{enumerate}
	\item{Badanie anharmoniczności drgań wahadła matematycznego}
	\item{Wyznaczanie przyspieszenia ziemskiego za pomocą wahadła różnicowego}
\end{enumerate}


\subsection{Wstęp teoretyczny}

\textbf{Wahadłem matematycznym płaskim} nazywamy punkt materialny, poruszający się
ruchem okrężnym w jednorodnym polu grawitacyjnym. Praktyczną realizacją tego pojęcia
jest obciążnik (najczęściej kulka) o małych rozmiarach, zamocowany do nierozciągliwej
nici o bardzo małej, pomijalnej masie (Rysunek \ref{wahadlo_matematyczne}). 

\begin{figure}[h]
\caption{Wahadło matematyczne}

\label{wahadlo_matematyczne}
\end{figure}

Siłę ciężkości $\vec{F}_g = m\vec{g}$ działającą na ciężarek możemy rozłożyć na
składową radialną $F_g \cos \theta$ oraz styczną do toru ruchu wahadła $F_g \sin \theta$.
Składowa styczna wpływa na powrót ciężarka do położenia równowagi, gdyż działa ona
zawsze przeciwnie do jego wychylenia. Z drugiej zasady dynamiki Newtona mamy więc:

\begin{align*}
 ma &= -mg\sin\theta \\
 \frac{d^2S}{dt^2} &= -g\sin\theta
\end{align*}

Ponieważ $S = l \cdot \theta$:

\[ \frac{d^2\theta}{dt^2} = -\frac{g}{l}\sin\theta \]

Nie istnieje analityczne rozwiązanie tego równania różniczkowego. Dla małych kątów
można uprościć powyższy wzór, zakładając, że \, $\sin\theta \approx \theta$ \,
-- wtedy równanie różniczkowe daje się rozwiązać analitycznie. W rezultacie otrzymujemy,
że \textbf{dla wahadła matematycznego poruszającego się w zakresie małych kątów}
(przy założeniu $\theta(0) = \theta_{max}$):

\[ \theta(t) = \theta_{max}\cos\left(\sqrt{\frac{g}{l}}\right), \] 

czyli ruch \textbf{jest harmoniczny}, a okres tego ruchu wynosi

\[ T = 2\pi~\sqrt{\frac{l}{g}} \]


W ogólności (bez założenia małych wychyleń), korzystając z rozwinięcia funkcji $sin$
w szereg, możemy finalnie otrzymać zależność okresu drgań wahadła $T$ od kąta
maksymalnego wychylenia $\theta_{max}$:

\[ 
	T = 2\pi ~ \sqrt[]{\frac{l}{g}}~f(\theta_{max}) 
, \enskip \text{gdzie} \enskip \enskip 
f(\theta_{max}) = 
	\sum\limits_{i=0}^{\infty} 
		\left[ \frac{(2n)!}{(2^nn!)^2} \right]^2 
			\sin^{2n}\left(\frac{\theta_{max}}{2} \right) \]

Ruch wahadła matematycznego jest więc w ogólności \textbf{anharmoniczny} -- okres
jest zależny od amplitudy.



\subsection{Układ pomiarowy}
Do przeprowadzenia obu części ćwiczenia posłużono się układem pomiarowym,
przedstawionym na rysunku \ref{uklad_pomiarowy}. Do ruchomego elementu 
połączonego z nieruchomym statywem przymocowana była długa, nierozciągliwa nitka, 
o znikomo małej masie, z zamocowanym na jej końcu metalowym obciążnikiem. 
Ruch elementu w pionie pozwalał modyfikować długość wahadła. Do statywu 
przytwierdzony był również kątomierz oraz linijka, służące do pomiaru kolejno:
kąta maksymalnego wychylenia wahadła oraz zmiany długości wahadła różnicowego.
Pod statywem znajdował się elektroniczny układ pomiarowy, złożony z fotokomórki
oraz modułu sterowania, za pomocą którego mierzony był okres wahadła.


\begin{figure}[b]
\caption{Schemat układu pomiarowego}

\label{uklad_pomiarowy}
\end{figure}



\end{document}