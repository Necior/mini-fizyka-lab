\documentclass[a4paper]{article}

%% Language and font encodings
\usepackage{polski}
\usepackage[utf8]{inputenc}
\usepackage[T1]{fontenc}
\usepackage{pdfpages}
\usepackage{indentfirst}
\usepackage{cancel}

% Adjust penalties
\brokenpenalty=1000
\clubpenalty=1000
\widowpenalty=1000

%% Sets page size and margins
\usepackage[a4paper]{geometry}

%% Useful packages
\usepackage{amsmath}
\usepackage{graphicx}
\usepackage[colorlinks=true, allcolors=blue]{hyperref}
\usepackage{booktabs}
\usepackage{ulem}
\usepackage{tikz}

\usepackage{float}

\renewcommand\thesection{\arabic{section}.}
\renewcommand\thesubsection{\arabic{section}.\arabic{subsection}.}
\renewcommand\thesubsubsection{\arabic{section}.\arabic{subsection}. \arabic{subsubsection}.}

% The following commands are not supported in PSTricks at present
% We define them conditionally, so when they are implemented,
% this pgf file will use them.
\ifx\du\undefined
  \newlength{\du}
\fi
\setlength{\du}{15\unitlength}

\newcommand{\Vsp}[1]{\vtop to #1 {}}
\newcommand{\Hsp}[1]{\hbox to #1 {}}
\newcommand{\Small}{\scriptsize}

\title{Sprawozdanie nr 5}
\date{}


\begin{document}

\begin{center}
\begin{tabular}{|p{5.5cm}|l|l|c|}
    \hline
	% Row 1.1  
	    Wydział \Vsp{4mm} &
	    \multicolumn{1}{|l}{Dzień} &
	    poniedziałek $17^{15} - 19^{30}$ &
	    Nr zespołu \\
	% Row 1.2
	    \mbox{\small{Matematyki i Nauk Informatycznych}} &
	    \multicolumn{1}{|l}{Data}  &
	    &
	    \multicolumn{1}{c|}{\Large{18}} \\
    
    \hline
	% Row 2.1 
	    Nazwisko i Imię: &
	    \Small Ocena z przygotowania &
	    \Small Ocena ze sprawozdania &
	    \Small Ocena Końcowa \\
	% Rows 2.2-2.4
	    1. Jasiński Bartosz & & &\\
	    2. Sadłocha Adrian & & & \\
	    3. Wódkiewicz Andrzej & & & \\

    \hline
    % Row 3.1
	    \multicolumn{2}{|l|}{Prowadzący \Vsp{4mm}} &
	    \multicolumn{2}{|l|}{Podpis prowadzącego} \\  
    % Row 3.2
    	\multicolumn{2}{|l|}{} &
    	\multicolumn{2}{|l|}{} \\    	
    \hline
\end{tabular}
\label{pieczatka}
\end{center}

{\let\newpage\relax\maketitle}  % stolen from: https://tex.stackexchange.com/questions/86249/maketitle-text-before-title
\setcounter{secnumdepth}{2}


\section{Opis ćwiczenia}
Ćwiczenie złożone było z następujących części:
\begin{enumerate}
	\item{Badanie prawa Malusa}
	\item{Badanie prawa Snella}
	\item{Wyznaczenie kąta granicznego}
	\item{Wyznaczenie kąta Brewstera}
\end{enumerate}

\subsection{Wstęp teoretyczny}
\paragraph*{Polaryzacja fali} to właściwość fali poprzecznej (m.in. elektromagnetycznej), polegająca na uporządkowaniu relacji między kierunkiem oscylacji zaburzenia a kierunkiem rozchodzenia się fali. \cite{wiki}
Wyróżniamy m.in. \textbf{polaryzację liniową}, przy której oscylacje fali zachodzą tylko w jednej płaszczyźnie, zawierającej kierunek rozchodzenia się fali

\paragraph*{Prawo Malusa} głosi, że gdy idealny polaryzator jest umiejscowiony na drodze spolaryzowanej wiązki światła, natężenie światła $I$ przechodzące przez ten polaryzator jest równe:
\[
	I = I_0 \cos^2 \theta
\]
gdzie $I_0$ to natężenie początkowe, a $\theta$ to kąt między kierunkiem polaryzacji światła a osią polaryzatora.

\paragraph*{Prawo Snella} głosi, że stosunek sinusów kątów padania i załamania jest wprost proporcjonalny do stosunku prędkości fazowych w tych ośrodkach oraz odwrotnie proporcjonalny do stosunku współczynników załamania tych ośrodków:
\[
	\frac{\sin \alpha}{\sin \beta} = \frac{v_1}{v_2} = \frac{n_2}{n_1}
\]

\paragraph*{Kąt graniczny} jest to maksymalny kąt, pod jakim światło może padać na granicę danych ośrodków, aby wciąż ulec załamaniu. Występuje on tylko w przypadku padania światła \textbf{z ośrodka o większym współczynniku załamania do ośrodka o mniejszym współczynniku}. Podstawiąjąc do wzoru Snella maksymalny kąt załamania o mierze $90^\circ$ otrzymujemy:
\begin{align*}
	\frac{\sin \alpha_{gr}}{\sin 90^\circ} &= \frac{n_2}{n_1} \\
	\alpha_{gr} &= \arcsin \left(\frac{n_2}{n_1}\right)
\end{align*}

\paragraph*{Kąt Brewstera} jest to kąt padania, przy którym wiązka światła odbitego jest całkowicie spolaryzowana liniowo. Zjawisko to zachodzi w przypadku, gdy promień odbity i załamany tworzą kąt prosty. 
Oscylacje wektora elektrycznego odbywają się zawsze prostopadle do kierunku ruchu fali -- fala nie może poruszać się w kierunku własnych drgań.
Kierunek ruchu światła odbitego jest prostopadły do kierunku ruchu światła załamanego.
Dlatego światło odbite nie może zawierać światła o drganiach w płaszczyźnie odbicia, ponieważ są one równoległe do kierunku ruchu światła załamanego.

Mamy więc:
\[ \alpha + \beta = 90^\circ \]
Z prawa Snella:
\[ 	\frac{\sin \alpha_{B}}{\sin \beta} = \frac{n_2}{n_1} \]
Ponieważ kąt padania = kąt odbicia: 
\begin{align*}
 	\frac{\sin \alpha_{B}}{\sin (90^\circ - \alpha_B)} &= \frac{n_2}{n_1} \\
	\frac{\sin \alpha_{B}}{\cos (\alpha_B)} &= \frac{n_2}{n_1} \\
	\text{tg} \ \alpha_{B} &= \frac{n_2}{n_1} \\
	\alpha_B &= \text{arctg} \left(\frac{n_2}{n_1}\right)
\end{align*}
\subsection{Układ pomiarowy}
Układ pomiarowy składał się z 2 polaryzatorów (nazywanych dalej \textit{polaryzatorem} i \textit{analizatorem}), lasera, fotodetektora z amperomierzem, stolika goniometrycznego oraz płytki z pleksiglasu o przekroju półkola w płaszczyźnie poziomej.


\section{Pomiary i obliczenia}
\subsection{Badanie prawa Malusa}
Przy pomocy obu polaryzatorów oraz fotodetektora została zmierzona wartość natężenia światła spolaryzowanego.
Wpierw odnaleziony został taki kąt obrotu analizatora, przy którym mierzona wartość natężenia światła była maksymalna ($\alpha_0 = 176^\circ$).
Następnie, siedmiokrotnie dokonano obrotu analizatora o $15^\circ$ i pomiaru wartości natężenia.
Wyniki zostały przedstawione w tablicy \ref{malus-pomiary}. 
% Pomiary 1. i 7. zostały uznane jako błędy grube na podstawie próby dopasowania wykresu funkcji $263 \cos^2(x+10)$ do wyników pomiarów (patrz rysunek \ref{malus-wykres}).
Na rysunku \ref{malus-wykres} przedstawione zostały 2 próby jak najlepszego dopasowania wykresu funkcji $\cos^2$ uwzględniając wszystkie niepewności standardowe, przy założeniach:
\begin{itemize}
	\item kąt $\alpha_0$ jest kątem, dla którego natężenie jest maksymalne -- kolor czerwony, linia ciągła
	\item kąty obrotu analizatora zostały odczytane z przesunięciem $10^\circ$, pomiar $k=1$ jest błędem grubym, a kąt $\alpha_0$ \textbf{nie} jest kątem maksymalnego natężenia -- kolor zielony, linia przerywana
\end{itemize}

Biorąc pod uwagę trudności podczas przeprowadzania ćwiczenia, pomimo większej ilości założeń prawdziwy zdaje się być przypadek drugi (linia przerywana na wykresie).


\begin{table}[h]
\centering
\begin{tabular}{rrrr}
\toprule
 $k$ &  $\alpha_k \ ({}^\circ)$ &  $I \ (\mu\text{A})$ &  $u_I \ (\mu \text{A})$ \\
\midrule
 0 &       176 &   260.0 &         3.662877 \\
 \textit{1} & 	   \textit{161} &   \textit{240.0} &         \textit{3.662877} \\
 2 &       146 &   225.0 &         3.662877 \\
 3 &       131 &   160.0 &         3.662877 \\
 4 &       116 &    94.0 &         1.414214 \\
 5 &       101 &    32.0 &         1.414214 \\
 6 &        86 &     2.2 &         0.036629 \\
 7 &        71 &    13.0 &         0.366288 \\
\bottomrule
\end{tabular}
\caption{Pomiary natężenia światła spolaryzowanego}
\label{malus-pomiary}
\end{table}

\begin{figure}[b]
\centering
\includegraphics[scale=0.8]{malus.png}
\caption{Prawo Malusa}
\label{malus-wykres}
\end{figure}

Wykres funkcji $266 \cdot \cos^2(\alpha + 9)$ (gdzie argument funkcji cosinus: $(\alpha + 9)$ jest podany w stopniach) zdaje się dobrze odzwierciedlać uzyskane wyniki pomiarów, potwierdzając tym samym, że natężenie światła spolaryzowanego liniowo jest proporcjonalne do kwadratu cosinusa kąta między płaszczyzną polaryzacji światła padającego a płaszczyzną polaryzacji światła po przejściu przez polaryzator.


\subsection{Badanie prawa Snella}
Następnie, wykonano pomiary prowadzące do wyznaczenia współczynnika załamania światła dla płytki z pleksiglasu.
W tym celu zmierzono kąty padania światła, jak i kąty odbicia.
Wyniki zostały przedstawione w tabeli \ref{snell-pomiary}.
Standardowa niepewność pomiarowa pomiaru kąta padania jak i kąta odbicia, została oszacowana na:
\[
	u(x) = \sqrt{ 
		\left( \frac{0.0175}{\sqrt{3}} \right)^2 
	  + \left( \frac{0.0009}{\sqrt{3}} \right)^2
	}
\approx 0.0114 \ \text{rad}
\]
Korzystając z metody propagacji błędu dla pomiarów pośrednich, jakimi są sinusy mierzonych kątów, otrzymujemy, że:
\[
	u(\sin(x)) = \sqrt{\left(\frac{\partial \sin(x)}{\partial x}u(x)\right)^2} = \left|\cos(x)u(x)\right|
\]

Pomiary wraz z niepewnościami na iksach i igrekach zostały naniesione na wykresie (rysunek \ref{snell-wykres}).
Następnie, korzystając z metody najmniejszych kwadratów, zostało odnalezione najlepsze dopasowanie funkcji liniowej, przedstawione na wykresie kolorem czerwonym.
W tym celu skorzystano z biblioteki \texttt{numpy} w języku Python, która pozwoliła obliczyć automatycznie $a$ i $b$ wraz z niepewnościami wg poniższych wzorów:
\begin{align*}
a &= \frac{n \Sigma X Y - \Sigma X \Sigma Y}{n \Sigma X^2 - \left(\Sigma X\right)^2} \\
b &= \frac{1}{n}\left(\Sigma Y - a \Sigma X\right) \\
u(a) &= \sqrt{\frac{n}{n-2} \frac{\Sigma Y^2 - a\Sigma XY - b\Sigma Y}{n\Sigma X^2-\left(\Sigma X \right)^2}} \\
u(b) &= u(a)\cdot \sqrt{\frac{\Sigma X^2}{n}}
\end{align*}
gdzie $X = \sin\alpha, \ Y = \sin\beta, \ n = 7$.


Współczynnik nachylenia prostej wyniósł $a = 0.66688$, a wyraz wolny: $b = 0.00415$.
Niepewności uzyskanych wartości to odpowiednio:
$ u_a=0.00591, \ u_b=0.00411 $

Zatem, korzystając z prawa Snella:
\begin{align*}
\frac{n_{plytka}}{n_{powietrze}} &= \frac{\sin\alpha}{\sin\beta} \\ 
\frac{n_{plytka}}{n_{powietrze}} &= \frac{\sin\alpha}{a \cdot \sin\alpha + b}
\end{align*}

Zakładając, że $n_{powietrze} \approx 1, \ b \approx 0$ otrzymujemy:
\[ n_{plytka} = \frac{1}{a}\]

Stąd: $n_{plytka} = \frac{1}{0.66688} \approx 1.49952$

Niepewność standardową otrzymanego wyniku otrzymamy ponownie ze wzoru na propagację błędu:

\[
		u(n_{plytka}) = \sqrt{\left(\frac{\partial n_{plytka}(a)}{\partial a}u_{a}\right)^2} = 
		\left| \frac{1}{a^2} u_{a}\right| = \frac{1}{0.66688^2}\cdot 0.00591 \approx 0.01329
\]

Zatem uwzględniając niepewność standardową, współczynnik załamania światła płytki wykorzystanej w ćwiczeniu (obliczony przy użyciu metody wykorzystującej prawo Snella) wynosi \[ n_{plytka} = 1.500(13)\]
\begin{table}[h!]
\centering
\begin{tabular}{lrrrrrrrr}
\toprule

l.p.&  $\alpha \ ({}^\circ)$ & $\beta \ ({}^\circ)$ &  $\alpha \ (\text{rad})$ &  $\beta \ (\text{rad})$ &  $\sin(\alpha)$ &  $\sin(\beta)$ &  $u(\sin(\alpha))$ &  $u(\sin(\beta))$ \\
\midrule
0 &           5 &         3.5 &    0.087266 &    0.061087 &   0.087156 &   0.061049 &  0.011317 &  0.011339 \\
1 &          15 &        10.5 &    0.261799 &    0.183260 &   0.258819 &   0.182236 &  0.010973 &  0.011170 \\
2 &          30 &        19.5 &    0.523599 &    0.340339 &   0.500000 &   0.333807 &  0.009838 &  0.010708 \\
3 &          45 &        28.0 &    0.785398 &    0.488692 &   0.707107 &   0.469472 &  0.008033 &  0.010030 \\
4 &          60 &        36.0 &    1.047198 &    0.628319 &   0.866025 &   0.587785 &  0.005680 &  0.009190 \\
5 &          70 &        39.0 &    1.221730 &    0.680678 &   0.939693 &   0.629320 &  0.003885 &  0.008828 \\
6 &          75 &        40.5 &    1.308997 &    0.706858 &   0.965926 &   0.649448 &  0.002940 &  0.008638 \\
\bottomrule
\end{tabular}

\caption{Pomiary zależności kąta odbicia ($\beta$) od kąta padania ($\alpha$)}
\label{snell-pomiary}
\end{table}


\begin{figure}
\centering
\includegraphics[scale=0.7]{snell.png}
\caption{Prawo Snella}
\label{snell-wykres}
\end{figure}


\subsection{Wyznaczenie kąta granicznego}
Kolejnym ćwiczeniem było wyznaczenie kąta granicznego zadanej przezroczystej płytki.
Obserwując załamaną wiązkę światła, przechodzącą przez granicę "szkło/powietrze", szukany był kąt $\alpha_{gr}$, powyżej którego wiązka światła zanikała.
Wartość tego kąta została wyznaczona eksperymentalnie na: 
\[
\alpha_{gr} = 43^\circ \approx 0.7504916 \ \text{rad}
\]
Standardowa niepewność pomiaru została oszacowana na 
\[
u(\alpha_{gr}) = \sqrt{ \left( \frac{0.0349}{\sqrt{3}} \right)^2 + \left( \frac{0.0175}{\sqrt{3}} \right)^2}
\approx 0.02254 \ \text{rad}
\]

Ponieważ z definicji kąta granicznego mamy: 
\[ 
	\frac{\sin \alpha_{gr}}{\sin 90^\circ} = \frac{n_{powietrze}}{n_{plytka}} ,
\]
a zakładamy, że
\[ 
	n_{powietrze} = 1,
\]
stąd:
\[
	n_{plytka} = \frac{1}{\sin \alpha_{gr}} \approx 1.4662792
\]
Propagacja niepewności:
\[
	u(n_{plytka}) = \sqrt{\left( \frac{\partial n_{plytka}}{\partial \alpha_{gr}}u(\alpha_{gr})\right)^2} = 
	\left| \frac{2\cos(\alpha_{gr})}{\cos(2\alpha_{gr})-1} \, u(\alpha_{gr}) \right| \approx 0.03544
\]
Zatem współczynnik załamania światła obliczony za pomocą wyznaczenia kąta granicznego wynosi:
\[
	n_{plytka} = 1.466(35)
\]

\subsection{Wyznaczenie kąta Brewstera}
Ostatnim ćwiczeniem było wyznaczenie kąta Brewstera.
W tym celu szukano takiego kąta padania światła, dla którego wiązka odbita będzie tworzyła wraz z wiązką załamaną kąt $90^\circ$.
Kąt ten został wyznaczony eksperymentalnie na $56^\circ$ stopni, a niepewność standardowa została oszacowana na \[
	u(\alpha_B) =\sqrt{ 
		  \left( \frac{0.0175}{\sqrt{3}} \right)^2 
		+ \left( \frac{0.0175}{\sqrt{3}} \right)^2
		+ \left( \frac{0.009}{\sqrt{3}} \right)^2
	}
	 \approx 0.015 \ \text{rad}
\]

Korzystając z zależności:
\[
	\text{tg}\alpha_B = \frac{n_{plytka}}{n_{powietrze}}
\]
i ponownie stosując podstawienie $n_{powietrze} = 1$, otrzymujemy:
\[
	n_{plytka} = \text{tg}56^\circ \approx 1.483
\]

Propagacja niepewności:
\[
	u(n_{plytka})
		= \sqrt{
			\left( \frac{\partial n_{plytka}}{\partial \alpha_B}u(\alpha_B)\right)^2
		}
		=	\left| \frac{1}{\cos^2 \alpha_B} \, u(\alpha_B) \right| 
		\approx 0.048
\]

Zatem współczynnik załamania światła obliczony za pomocą wyznaczenia kąta Brewstera wynosi:
\[
	n_{plytka} = 1.483(48)
\]
\subsection{Wnioski}
Pierwsza część ćwiczenia pozwoliła nam pokazać, że zgodnie z prawem Malusa, natężenie światła spolaryzowanego liniowo, przechodzącego przez analizator jest wprost proporcjonalne do kwadratu cosinusa kąta między płaszczyzną polaryzacji światła wpadającego a płaszczyzną polaryzacji światła wychodzącego z analizatora.

Kolejne trzy części ćwiczenia miały na celu wyznaczenie współczynnika liniowego płytki z pleksiglasu na 3 różne sposoby.
Wyniki zostały przedstawione w tabeli \ref{wnioski-wartości}.

\begin{table}[h!]
\centering
\begin{tabular}{lrr}
\toprule
Metoda & $n_{plytka}$ & $u(n_{plytka})$ \\
\midrule
Prawo Snella  & 1.500 & 0.013 \\
Kąt graniczny & 1.466 & 0.035 \\
Kąt Brewstera & 1.483 & 0.048 \\
\bottomrule
\end{tabular}
\caption{Uzyskane wartości współczynnika załamania światła dla płytki z pleksiglasu}
\label{wnioski-wartości}
\end{table}

Korzystając z wartości współczynnika załamania światła dla pleksiglasu, wyznaczonej przez Cargille Labs \cite{cargille}, tj. $n = 1.4917$, możemy porównać uzyskane wyniki.
Wykres uzyskanych wyników wraz z ich niepewnością rozszerzoną przy $k = 1$ przedstawiony został na rysunku \ref{wnioski-wykres}.

Można zauważyć, że wszystkie metody pozwoliły na otrzymanie poprawnych wyników (w granicach ich niepewności).
Metodą z najmniejszą niepewnością pomiarową okazała się metoda wykorzystująca prawo Snella.
Przyczyną tego było wykonanie \textbf{kilku} pomiarów (w odróżnieniu do pozostałych metod) i wykorzystanie metody najmniejszych kwadratów.

Aby jeszcze bardziej zminimalizować niepewność, należałoby wykonać więcej pomiarów dla metody wykorzystującej prawo Snella oraz dla wszystkich trzech metod postarać się o lepszą kalibrację stolika goniometrycznego, jak i zaproponować lepszy sposób na odczyt wartości kątów pod jakimi padają wiązki światła.

\begin{figure}
\centering
\includegraphics[scale=0.7]{wnioski.png}	
\caption{Porównanie metod wyznaczania wsp. załamania światła}
\label{wnioski-wykres}
\end{figure}




\begin{thebibliography}{9}

\bibitem{wiki}
  Polaryzacja fali -- artykuł na Wikipedii
  
  https://pl.wikipedia.org/wiki/Polaryzacja\_fali

\bibitem{cargille}
  Strona internetowa firmy Cargille-Sacher Laboratories Inc,
  
  http://www.cargille.com/specimmerliq.shtml


\end{thebibliography}

\end{document}
